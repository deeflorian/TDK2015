\chapter{Background}
\label{chap:background}

The design of complex cyber-physical systems is an interdisciplinary process. From the engineering point of view, defining the requirements, designing sufficiently reliable components, dealing with scalability issues, \textbf{constructing} (jobb igy?) testing processes that result in high test coverage, verifying safety critical components, and maintainability issues are all present at the same time. Systems engineering focuses on how to design and manage such systems \cite{randomwikipedialink1} \cite{randomwikipedialink2}. 
%Models for sw development
%Need for model driven development
%Need for formal verifiability
%Need for runtime verifiability
\section{Model driven software development}
Model driven software development (MDSD) emphasises problem solving by the development and maintenance of models describing the system being designed. MDSD heavily relies on automated code and documentation generation based on the models of components or the overall model of the system. Modelling has the advantage of introducing abstractions, thus reducing the complexity of the development process. The dynamic code generation guarantees that the code will inherit the properties that can be directly derived from the model, while reducing the costs by eliminating unnecessary round-trip engineering. Model based approaches also have the advantage of easier testability, or if the model is formal enough they can make verification possible.
\section{Modelling methods}
Various methods and tools are available for generating test cases and monitoring components, and formally verifying certain properties of models. Even though certain model transformations are possible, the available tools... To be continued. Have to fix my chapter first... ~agonising batman picture~
  \subsection{MDD method 1}
  \subsection{MDD method 2}
  \subsection{MDD method 3}
