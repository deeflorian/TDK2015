\chapter{Conclusion}
\label{chap:conclusion}

Cyber physical and safety critical systems are difficult to analyse at design time, so various techniqes need to be applied during the development and operation. In our work, we focus on the analysis of such systems during their operation.

In this report we introduce the concept of hierarchical runtime verification to extend the traditional model driven development methodology with a model driven runtime verification approach.

Our framework combines the advantages of various modeling languages and provide the engineers with the ability to use familiar modeling formalisms to design the properties to be analysed runtime. In addition, our approach supports the hierarchical composition of the various runtime verification tasks to provide system level assurance of safety properties. 

For a proof of concept implementation, we integrated statechart based component level runtime verification with a complex event processing based high level formalims. In our work we also aimed to eliminate the disadvantages of the approaches used at different levels of verification. The low level componenswise monitoring is supported with a well-established statechart formalism, which is also formalised with the help of transition systems. This formalization enables us to extend our framework and provide the formal analysis of the runtime verification properties. To take a further step in this direction, an intermediate formal language is proposed to be under the high level (system level) engineering modeling language. The intermediate language supports also the further extension of the framework with other high level modeling languages. 

We illustrated the applicability of our approach with the help of a case-study. In addition, our framework is used for educational purposes to illustrate runtime verification at the courses of the department.