\chapter{Introduction}
\label{chap:introduction}

According to industrial estimates, the number of various smart devices - communicating with either us or each other - will raise to 50 billion, forming one of the most complex systems on the world wide web. This network of nearly unlimited computing power will not only consist of simple sensors and mobile phones, but also cars, airplanes, and medical devices on which lives depend upon. Cyber-physical systems open up radically new opportunities: accidents can be avoided by cars communicating with each other, and the energy consumption of smart buildings can be drastically lower, just to name a few.

The traditional critical embedded systems often use runtime verification with the goal of synthesizing monitoring programs to discover faulty components, whose behaviour differ from that of the specification.

The computing and data-processing capabilities of cyber-physical systems, coupled with their sensors and actuators make it possible to create a hierarchical, layered structure of high-reliability monitoring components with various responsibilities. The traditional critical components' high-reliability monitors' responsibilities can be limited to a local scope. This allows the creation of system-level monitors based on independent and global sensory data. These monitors are less reliable, but can predict errors in earlier stages.

This paper describes a hierarchical, mathematically precise, runtime verification framework which supports (1) the critical components' monitors automatic synthetisation from a high-level statechart formalism, (2) as well as the creation of hierarchical, system-level monitors based on complex event-processing. The results are presented as a case study of the monitoring system of a model railway track, where collisions are avoided by using multi-level runtime verification.

\newpage