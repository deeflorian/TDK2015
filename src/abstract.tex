\addcontentsline{toc}{chapter}{Abstract}

\begin{otherlanguage}{magyar}
\thispagestyle{plain}

\paragraph*{Összefoglalás}

Ipari becslések szerint 2020-ra 50 milliárdra nő a különféle okoseszközök száma, amelyek egymással és velünk kommunikálva komplex rendszert alkotnak a világhálón. A szinte korlátlan kapacitású számítási felhőbe azonban az egyszerű szenzorok és mobiltelefonok mellett azok a kritikus beágyazott rendszerek – autók, repülőgépek, gyógyászati berendezések - is bekapcsolódnak, amelyek működésén emberéletek múlnak. A kiberfizikai rendszerek radikálisan új lehetőségeket teremtenek: az egymással kommunikáló autók baleseteket előzhetnek meg, az intelligens épületek energiafogyasztása csökken.

A hagyományos kritikus beágyazott rendszerekben gyakorta alkalmazott módszer a futási idejű ellenőrzés. Ennek célja olyan ellenőrző programok szintézise, melyek segítségével felderíthető egy kritikus komponens hibás, a követelményektől eltérő viselkedése a rendszer működése közben.

Kiberfizikai rendszerekben a rendelkezésre álló számítási felhő adatfeldolgozó kapacitása, illetve a különféle szenzorok és beavatkozók lehetővé teszik, hogy több, egymásra hierarchikusan épülő, különböző megbízhatóságú és felelősségű ellenőrzési kört is megvalósíthassunk. Ennek értelmében a hagyományos, kritikus komponensek nagy megbízhatóságú monitorai lokális felelősségi körben működhetnek. Mindezek fölé (független és globális szenzoradatokra építve) olyan rendszerszintű monitorok is megalkothatók, amelyek ugyan kevésbé megbízhatóak, de a rendszerszintű hibát prediktíven, korábbi fázisban detektálhatják.

A TDK dolgozatban egy ilyen hierarchikus futási idejű ellenőrzést támogató, matematikailag precíz keretrendszert dolgoztunk ki, amely támogatja (1) a kritikus komponensek monitorainak automatikus szintézisét egy magasszintű állapotgép alapú formalizmusból kiindulva, (2) valamint rendszerszintű hierarchikus monitorok létrehozását komplex eseményfeldolgozás segítségével. A dolgozat eredményeit modellvasutak monitorozásának (valós terepasztalon is megvalósított) esettanulmányán keresztül demonstráljuk, amely többszintű ellenőrzés segítségével képes elkerülni a vonatok összeütközését.
\end{otherlanguage}